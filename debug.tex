Of course. Here is a clear, best-practice project structure for your
Java file conversion terminal app, presented as a tree diagram. This
structure emphasizes separation of concerns, making your project
scalable, maintainable, and easy to navigate.

\begin{verbatim}
file-converter-app/
├── .gitignore
├── pom.xml
├── README.md
└── src/
    ├── main/
    │   ├── java/
    │   │   └── com/
    │   │       └── yourcompany/
    │   │           └── fileconverter/
    │   │               ├── Main.java
    │   │               ├── cli/
    │   │               │   ├── ConvertCommand.java
    │   │               │   └── SupportedFormatsCommand.java
    │   │               ├── conversion/
    │   │               │   ├── ConversionService.java
    │   │               │   ├── converters/
    │   │               │   │   ├── DocumentConverter.java
    │   │               │   │   ├── ImageConverter.java
    │   │               │   │   └── MultimediaConverter.java
    │   │               │   └── TikaFileTypeDetector.java
    │   │               ├── exception/
    │   │               │   ├── ConversionException.java
    │   │               │   └── UnsupportedFileTypeException.java
    │   │               └── util/
    │   │                   └── FileSystemUtil.java
    │   └── resources/
    │       └── log4j2.xml
    └── test/
        ├── java/
        │   └── com/
        │       └── yourcompany/
        │           └── fileconverter/
        │               ├── cli/
        │               │   └── ConvertCommandTest.java
        │               └── conversion/
        │                   ├── TikaFileTypeDetectorTest.java
        │                   └── converters/
        │                       └── DocumentConverterTest.java
        └── resources/
            ├── documents/
            │   └── sample.md
            └── images/
                └── sample.png
\end{verbatim}

\subsubsection{\texorpdfstring{\textbf{Explanation of the
Structure:}}{Explanation of the Structure:}}\label{explanation-of-the-structure}

\begin{itemize}
\item
  \textbf{\texttt{file-converter-app/}}: The root directory of your
  project.

  \begin{itemize}
  \tightlist
  \item
    \textbf{\texttt{.gitignore}}: Specifies intentionally untracked
    files to ignore (e.g., \texttt{target/}, \texttt{.idea/},
    \texttt{*.log}).
  \item
    \textbf{\texttt{pom.xml}}: Your Maven Project Object Model file. It
    defines project dependencies, plugins, and build settings.
  \item
    \textbf{\texttt{README.md}}: Essential documentation for your
    project. Include a description, installation instructions, and usage
    examples.
  \end{itemize}
\item
  \textbf{\texttt{src/main/java/com/yourcompany/fileconverter/}}: The
  base package for your Java source code, following standard Java
  package naming conventions.

  \begin{itemize}
  \item
    \textbf{\texttt{Main.java}}: The main entry point of the
    application. Its primary role is to initialize Picocli and execute
    the appropriate command.
  \item
    \textbf{\texttt{cli/}}: This package holds all your command-line
    interface classes, powered by Picocli.

    \begin{itemize}
    \tightlist
    \item
      \textbf{\texttt{ConvertCommand.java}}: The main command logic for
      the \texttt{convert} action. It will handle command-line options
      like input file, output file, and format.
    \item
      \textbf{\texttt{SupportedFormatsCommand.java}}: A potential
      subcommand (e.g., \texttt{convert\ formats}) to list all supported
      input and output file types.
    \end{itemize}
  \item
    \textbf{\texttt{conversion/}}: The core logic for the file
    conversion process resides here.

    \begin{itemize}
    \tightlist
    \item
      \textbf{\texttt{ConversionService.java}}: A central service that
      orchestrates the conversion. It takes the input file, determines
      its type, and delegates the conversion task to the appropriate
      converter.
    \item
      \textbf{\texttt{TikaFileTypeDetector.java}}: A dedicated class
      that uses Apache Tika to accurately detect the MIME type of a
      file.
    \item
      \textbf{\texttt{converters/}}: This sub-package contains specific
      converter implementations for different file categories.

      \begin{itemize}
      \tightlist
      \item
        \textbf{\texttt{DocumentConverter.java}}: Handles conversions
        for document types (e.g., Markdown to PDF) by interfacing with
        Pandoc.
      \item
        \textbf{\texttt{ImageConverter.java}}: Manages image format
        conversions (e.g., PNG to JPG) perhaps using a library like
        \texttt{ImageIO}.
      \item
        \textbf{\texttt{MultimediaConverter.java}}: Orchestrates audio
        and video conversions by building and executing FFmpeg commands.
      \end{itemize}
    \end{itemize}
  \item
    \textbf{\texttt{exception/}}: Defines custom exceptions for better
    error handling.

    \begin{itemize}
    \tightlist
    \item
      \textbf{\texttt{ConversionException.java}}: A generic exception
      for errors that occur during the file conversion process.
    \item
      \textbf{\texttt{UnsupportedFileTypeException.java}}: Thrown when
      the application encounters a file type it cannot handle.
    \end{itemize}
  \item
    \textbf{\texttt{util/}}: A package for helper classes and utility
    functions.

    \begin{itemize}
    \tightlist
    \item
      \textbf{\texttt{FileSystemUtil.java}}: Contains helper methods for
      file operations, such as validating paths to prevent security
      vulnerabilities (e.g., directory traversal) or ensuring output
      directories exist.
    \end{itemize}
  \end{itemize}
\item
  \textbf{\texttt{src/main/resources/}}: This directory is for non-code
  resources that are bundled with your application.

  \begin{itemize}
  \tightlist
  \item
    \textbf{\texttt{log4j2.xml}}: Configuration file for a logging
    framework like Log4j 2, allowing for flexible and powerful logging
    instead of just \texttt{System.out.println()}.
  \end{itemize}
\item
  \textbf{\texttt{src/test/}}: Contains all your test code, mirroring
  the structure of your main source code.

  \begin{itemize}
  \tightlist
  \item
    \textbf{\texttt{java/}}: The root for your test classes.
  \item
    \textbf{\texttt{resources/}}: Test-specific resources, such as
    sample files (\texttt{sample.md}, \texttt{sample.png}) needed to run
    your unit and integration tests.
  \end{itemize}
\end{itemize}

CChecklist: -Dependency Checking: Programmatically check if Pandoc and
FFmpeg are installed. -Image Conversion: Adding a new converter for JPG,
PNG, etc. -Flexible Output Format: Adding a --format flag so a user can
convert video.mp4 to video.webm without typing the full output name.
-Dependency Checking: The professional next step. Programmatically check
if Pandoc and FFmpeg are installed and give helpful errors if they're
not. -Image Conversion: Add support for another major file category by
creating an ImageConverter. -Configuration File: Allow users to set
personal defaults (e.g., ``my default PDF margin is always 1.25in'').
